\section*{Appendix: Empirical SSNR model}

To develop empirical SSNR (smoothed SNR) models, we performed several
basic experiments in our testbed.  We transmitted a constant rate UDP
stream at 1 Mbps between two nodes. Then we changed the location of
the receiver and measured the value of SSNR and throughput.
Fig.\ref{fig:thruput} shows the results. We found that the values of
SSNR larger than 10 dB were stable enough for the threshold value
($S_{max}$).  In these experiments we used the WaveLAN NICs.

Next, we traced the relationship between SSNR and the communication
distance in a wireless outdoor environment. This experiment was
performed in our flat rectangular campus ground (300m x 300m) with no
obstacles or walls.  Fig.\ref{fig:snr-dis} shows the transition of
SSNR as a function of the distance. At small distance ($<$ 10m), we
have obtained the significantly increasing values of SSNR.  As we can
see from the figure, the values of SSNR increase considerably when the
distance is smaller than 10m. In addition, even when the distance is
larger than 10m, the values of SSNR monotonically decrease with
respect to the distance.  In this case, we used the IEEE 802.11b NICs.

\begin{figure}[htb]%%%%%%%%%%%%%%%%%%%%%%%%%%%%%%%%%%%%%%%%%%%
\begin{center}
\epsfile{file=image/thr-ssnr.eps,height=160pt}
\vspace{-6mm}
\caption{SSNR vs. throughput}
\label{fig:thruput}
\end{center}
\vspace{4mm}
\begin{center}
\epsfile{file=image/snr-dis.eps,height=130pt}
\caption{\label{fig:snr-dis} SSNR vs. Distance}
\end{center}
\end{figure}%%%%%%%%%%%%%%%%%%%%%%%%%%%%%%%%%%%%%%%%%%%

