\chapter{Implementation of HUSTLE}\label{chap:implementation}
%\section{introduction}\label{sec:implmentation_intro}
In this chapter we describe the implementation of HUSTLE. Firstly, we discuss about hardware and software in HUSTLE. In the next section, we describe the implementation of each module. Lastly, we discuss interactions in HUSTLE and summary this chapter.
\clearpage
\section{Hardware and Software}\label{sec:implementation_hardware}
To implement HUSTLE, we used Samsung Google Nexus S Smartphone, SunSpot Java Developer Kit that is shown in Figure \ref{fig:implement_hardware}. We used a photocell CDS 11mm MI11516 and a 3.5mm microphone jack as an accessory.
The full specification of Nexus S and Sunspot Java Developer Kit is shown in Table \ref{tab:implement_specific_sunspot} and Table \ref{tab:implement_specific_nexus}.

\begin{figure}[tb]
\centering
\includegraphics[width=0.9\textwidth]{image/eps/implement_hardware.eps}
\caption{Implementation: Hardware}
\label{fig:implement_hardware}
\end{figure}

\begin{table}[ptb]
\caption{Specific hardware of SunSpot Sensor Node}
\label{tab:implement_specific_sunspot}
\begin{center}
\begin{tabular}{|l|p{9cm}|}
\hline
\multirow{3}{*}{Processing} & 180 MHz 32 bit ARM920T core - 512K RAM - 4M Flash\\
					 &2.4 GHz IEEE 802.15.4 radio with integrated antenna\\
					 &AT91 timer chip\\ 
\hline					  
\multirow{7}{*}{Sensor Board}&2G/6G three-axis accelerometer\\
						&Temperature sensor\\
						&Light sensor\\
						&8 tri-color LEDs\\
						&6 analog inputs\\
						&2 momentary switches\\
						&5 general purpose I/O pins and 4 high current output pins\\
\hline
\multirow{7}{*}{Battery}&3.7V rechargeable 750 mAh lithium-ion battery\\
					&30 uA deep sleep mode\\
					&Automatic battery management provided by the software\\
\hline
\end{tabular}
\end{center}

\caption{Specific hardware of Samsung Galaxy Nexus S}
\label{tab:implement_specific_nexus}
\begin{center}
\begin{tabular}{|l|p{12cm}|}
\hline
\multirow{3}{*}{Processing} & Samsung Exynos 3110\\
					  &1 GHz ARM Cortex A8 based CPU core with a PowerVR SGX 540 GPU\\
					  &512MB of RAM - 16GB of NAND memory\\ 
\hline					  
\multirow{1}{*}{Screen}&4.0 inch touch screen\\
\hline
\multirow{9}{*}{Data input}&3-axis gyroscope\\
					&Accelerometer\\
					&Ambient light sensor\\
					&Capacitive touch-sensitive buttons\\
					&Digital compass\\
					&Microphone\\
					&Multi-touch capacitive touchscreen\\
					&Proximity sensor\\
					&Push buttons\\		
\hline
\multirow{2}{*}{Camera}&5.0 megapixel rear camera with LED flash\\
				    &VGA front camera\\
\hline
\end{tabular}
\end{center}

\end{table}%


With Nexus S Smartphone, we have a touch screen and a controllable flashlight. Nexus S is programmed by the Java language with Android SDK. In this time, we used Android SDK 4.1 to program it. Sunspot Java Developer Kit consist SunSpot sensor node and Base Station that is used as Sink Node. SunSPOT Java Developer Kit is also programmed by Java language with Sun SPOT SDK.

We use SunSPOT sensor nodes to make a secured WSN with multi-hop routing protocol and Network-wide keys scheme\cite{Simplicio:2010:SKM:1862461.1862545} for secure protocol.

\section{Modules Implementation}

This section describes implementation of each module in HUSTLE: user interface module, wireless communication module, controlling module, and light communication module.

\subsection{User interface module}

\begin{figure}[tb]
\centering
\includegraphics[width=0.9\textwidth]{image/eps/implementation_interface.eps}
\caption{User interfaces}
\label{fig:implementation_interface}
\end{figure}

This module plays a role in interacting with users such as how to start sending or receiving data and how to feedback to users. The interfaces are shown in Picture \ref{fig:implementation_interface}. We have four screens: (a) logging screen, (b) receiving light data for touch interface, (c) sending data with flashlight for blink interface and (d) feedback screen. To begin with, users must to login with their accounts. After that, to add a new sensor node, users can use (b) or (c) (Picture \ref{fig:implementation_interface}) which depends on hardware of sensor node. Lastly, in order to feedback to users, we show a list of added sensor node's information such as its type and address like Picture \ref{fig:implementation_interface}(d). Picture (b) shows interface in case of adding a new sensor node with accessory. We show light signals and also show status of reading data process. In Picture (c), we show interface of adding processing by using the flashlight of smartphone. We also show capture of camera, which can help users know what sensor nodes will be added.

\subsection{Wireless communication module}
This module is used to send data to others by using radio communication such as secure protocol data, routing data, or identification. This module receives commands from controlling module consist which and where data must be sent. In case address of destination is undefined, this module broadcasts the message to all sensor nodes.

\subsection{Controlling module}

\begin{table}[hp]
\caption{Rule table of Smartphone}
\label{tab:implement_rule_smartphone}
\begin{center}
\begin{tabular}{|p{5cm}|p{8cm}|}
\hline
Event&Action\\
\hline
{New light data}&{Send received key to Sink node}\\
\hline
{New added node information}&{Show it in screen}\\
\hline
{Push start button}& {Make a random security key and send it by flashlight}\\
\hline
\end{tabular}
\end{center}

\caption{Rule table of Sensor node with light sensor}
\label{tab:implement_rule_node_sensor}
\begin{center}
\begin{tabular}{|p{5cm}|p{8cm}|}
\hline
Event&Action\\
\hline
{New light data}&{Save received key}\\
\hline
{New information from sink node}&{Decrypt by saved key and setup with this data, send hello message to sink node}\\
\hline
\end{tabular}
\end{center}

\caption{Rule table of Sensor node with LED}
\label{tab:implement_rule_node_led}
\begin{center}
\begin{tabular}{|p{5cm}|p{8cm}|}
\hline
Event&Action\\
\hline
{Started}&{Make and save a random security key and send it by LED}\\
\hline
{New information from sink node}&{Decrypt by saved key and setup with this data, send hello message to sink node}\\
\hline
\end{tabular}
\end{center}

\caption{Rule table of Sink node}
\label{tab:implement_rule_sink}
\begin{center}
\begin{tabular}{|p{5cm}|p{8cm}|}
\hline
Event&Action\\
\hline
{New security key message}&{Encrypt setup data with this key and broadcast it without secure protocol}\\
\hline
{New hello message}&{Send this node information to Smartphone}\\
\hline
\end{tabular}
\end{center}

\end{table}%

This module determines which data will be sent when having an event such as new light data and new radio data. Rules table of each device is shown in Table \ref{tab:implement_rule_smartphone} \ref{tab:implement_rule_node_sensor} \ref{tab:implement_rule_node_led} \ref{tab:implement_rule_sink}.

\subsection{Light communication module}
This module is the most important one in our research, so we describe it more specific.
We split this module into 2 parts: encoding and decoding light pattern.

\subsubsection{Encoding Light Pattern}

This part is very simple, it only get string bits of data and send the light pattern according bit value. The Java code of this part is shown in Listing\ref{code:send_light_pattern}.

\lstinputlisting[label=code:send_light_pattern,caption=Encoding light pattern]{contents/source_code/send_light_pattern.java}

\subsubsection{Decoding Light Pattern: light sensor of sensor node}

As mentioned in the previous chapter (HUSLTE chapter), we have 3 steps: (1) getting light signals,(2) detecting period length time, (3) decoding period length time to light pattern and getting data. At first, we get light signals by using the light sensor at step (1) based on a window with size W\_LENGTH as Listing \ref{code:get_light_pattern} and Figure \ref{fig:hustle_one_window}.

\begin{figure}
\centering
\includegraphics[width=0.9\textwidth]{graph/eps/hustle_one_window.eps}
\caption{Decoding light pattern with light sensor: a window}
\label{fig:hustle_one_window}
\end{figure}

\lstinputlisting[label=code:get_light_pattern,caption=Light sensor case: read sensor data]{contents/source_code/get_light_signal.java}

In new thread, we calculate threshold then using this value to determine when the light period is starts or finishes. The source code of this task is shown in Listing \ref{code:dynamic_threshold} and \ref{code:detect_period}.

\lstinputlisting[label=code:dynamic_threshold,caption=Light sensor case: Dynamic threshold function]{contents/source_code/dynamic_threshold.java}

\lstinputlisting[label=code:detect_period,caption=Light sensor case: Detect light period function]{contents/source_code/detect_light_period.java}

After that, with light period information (start time and finish time) we can check light pattern such as START, FINISH ,0 bit or 1 bit; and we also get data from light pattern. This task is shown in Listing \ref{code:decode_period}.

\lstinputlisting[label=code:decode_period,caption=Light sensor case: Decoding light period function]{contents/source_code/decode_light_period.java}

\subsubsection{Decoding Light Pattern: photocell of accessory with Smartphone}

\begin{figure}
\centering
\includegraphics[width=0.9\textwidth]{graph/eps/hustle_decode_photon.eps}
\caption{Decoding light pattern with photocell: a window}
\label{fig:hustle_decode_photon}
\end{figure}

As said in HUSTLE chapter, in this case, we have only a difference on detecting light period step. We must detect peaks and check this peak whether it is true peak or noise peak. Source code for this step is shown in Listing \ref{code:detect_period2}

\lstinputlisting[label=code:detect_period2,caption=Photocell case: Detect light period function]{contents/source_code/detect_light_period_2.java}

\section{Interactions}

As mentioned in the previous chapter, HUSTLE provides two interactions for adding new sensor nodes, which is shown in Figure \ref{fig:implementation_interactions}. In touch interaction, users only have to turn on the new sensor node, touch it with the accessory of smartphone like Figure \ref{fig:implementation_interactions}A. In difference with touch interaction, in blink interaction (Figure \ref{fig:implementation_interactions}B) users can add multi-sensor nodes at the same time in one-shot. Users only have to turn on all new sensor nodes and direct flashlight of smartphone to them, which is used to send information.

\begin{figure}[hbtp]
\centering
\includegraphics[width=0.9\textwidth]{image/eps/implementation_interaction.eps}
\caption{Interactions in HUSTLE}
\label{fig:implementation_interactions}
\end{figure}

\section{Summary}
This section summarizes this chapter. Firstly, we described that we implement HUSTLE on Samsung Galaxy Nexus S Smartphone, SunSpot Java Development Kits. We also described implementation of each module such as controlling module, wireless communication module, light communication module and user interface module. At last, we showed the implementation of light communication module, encoding and decoding part with Java code, which is used in Smartphone and Sensor Node.

